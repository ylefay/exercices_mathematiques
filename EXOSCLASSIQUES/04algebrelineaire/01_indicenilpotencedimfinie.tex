\documentclass{article}
\usepackage[utf8]{inputenc}
\usepackage{amsmath}
\usepackage{amsmath}
\usepackage{amssymb}
\usepackage{french}
\usepackage{stmaryrd}
\usepackage{geometry}
\geometry{hmargin=2.5cm,vmargin=1.5cm}
\newcommand*{\QED}{\hfill\ensuremath{\blacksquare}}%
\begin{document}
\title{FICHE 0X-01 : Majoration d'indice de nilpotence : ALG? K-29-1-14}
\author{Yvann Le Fay}
\date{Juin 2019}
\maketitle

\section*{Enoncé}
Soit $E$ un espace vectoriel de dimension $n$, soit $f\in\mathcal{L}(E)$, nilpotente de rang $p_0$. Soit $g\in\mathcal{L}(E)$ telle que, pour tout $x\in E$, il existe $p_x\in\mathbb{N}^*$ telle que $f^{p_x}(x) = 0$
\begin{enumerate}
\item Majorer $p_0$. 
\item Montrer que $g$ est nilpotente.
\end{enumerate}

\section*{Solution}
\begin{enumerate}
\item Il existe $x_0\in E$ tel que pour tout $j\in\llbracket 0;p-1\rrbracket$, $f^j(x_0)$, cela par minimalité de $p_0$. On montre par récurrence immédiate que la famille $(x_0,f(x_0),\ldots, f^{p_0-1}(x_0))$ est une famille libre. On obtient donc $p_0\leq n$.
\item Soit $(e_1,\ldots,e_n)$ une base de $E$. L'indice de nilpotence de $g$ est alors $\max (p_{e_1},\ldots, p_{e_n})$.
\end{enumerate} \QED
\end{document}