\documentclass{article}
\usepackage[utf8]{inputenc}
\usepackage{amsmath}
\usepackage{amsmath}
\usepackage{amssymb}
\usepackage{french}
\usepackage{stmaryrd}
\usepackage{geometry}
\geometry{hmargin=2.5cm,vmargin=1.5cm}
\newcommand*{\QED}{\hfill\ensuremath{\blacksquare}}%
\begin{document}
\title{FICHE 01-01 : Nombre de dérangements : ALG1-02 1.2}
\author{Yvann Le Fay}
\date{Juin 2019}
\maketitle
\section{Première démonstration}
On note $D_k$ le nombre de dérangements à $k$-éléments.
Soit $p\in \llbracket 0;n\rrbracket$, 
\begin{align*}
\sum_{k=0}^{p}(-1)^k \underbrace{\begin{pmatrix}n\\k\end{pmatrix}\begin{pmatrix} n-k\\p-k\end{pmatrix}}_{=\begin{pmatrix}n\\p\end{pmatrix}} = \left\{
     \begin{array}{@{}l@{\thinspace}l}
      1 \textup{ si } p = 0\\
      0 \textup{ sinon }
     \end{array}
   \right.   
\end{align*}


On a la partition suivante
\begin{align*}
\mathfrak{S}_n = \bigsqcup_{k=0}^{n}{\{\sigma\in\mathfrak{S}_n : |\textup{supp}(\sigma)| = k \}}
\end{align*}

D'où (le $k$ devient $n-k$ par la symétrie du coefficient binomial)
\begin{align*}
n ! = \sum_{k=0}^{n} \begin{pmatrix}n\\k\end{pmatrix}D_k
\end{align*}

On calcule 
\begin{align*}
n!\sum_{k=0}^n \frac{(-1)^k}{k!} &= \sum_{k=0}^n (-1)^k \begin{pmatrix}n\\k\end{pmatrix} (n-k)!\\
&=\sum_{k=0}^n (-1)^k \begin{pmatrix} n\\k\end{pmatrix}\sum_{p=0}^{n-k}\begin{pmatrix}n-k\\p\end{pmatrix}D_p\\
&=\sum_{p=0}^n \sum_{k=0}^{n-p}(-1)^k \begin{pmatrix}
n\\k
\end{pmatrix} \begin{pmatrix}
n-k\\p
\end{pmatrix}D_p = D_n
\end{align*} \QED
\section{Seconde démonstration, préférée}
Posons pour tout $i\in\llbracket 1;n\rrbracket$, l'ensemble 
\begin{align*}
U_i = \{\sigma \in \mathfrak{S}_n : \sigma(i) = i\}
\end{align*}

Alors
\begin{align*}
D_n &= n!-\bigg|\bigcup_{i=1}^n U_i\bigg|\\
&= n!-\sum_{k=1}^n (-1)^{k-1}\sum_{1\leq i_1<\ldots<i_k\leq n}\bigg|\bigcap_{j=1}^{k}U_{i_j}\bigg|\\
&= n!-\sum_{k=1}^n (-1)^{k-1}(n-k)!\begin{pmatrix}
n\\k
\end{pmatrix}\\
&=n!\sum_{k=0}^n \frac{(-1)^k}{k!}
\end{align*}
\QED
\end{document}
