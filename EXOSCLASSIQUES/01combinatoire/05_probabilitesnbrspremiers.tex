\documentclass{article}
\usepackage[utf8]{inputenc}
\usepackage{amsmath}
\usepackage{amsmath}
\usepackage{amssymb}
\usepackage{french}
\usepackage{stmaryrd}
\usepackage{geometry}
\geometry{hmargin=2.5cm,vmargin=1.5cm}
\newcommand*{\QED}{\hfill\ensuremath{\blacksquare}}%
\begin{document}
\title{FICHE 02-05 : Probabilité que $k$ nombres soient premiers entre eux : ALG1-01 ?}
\author{Yvann Le Fay}
\date{Juin 2019}
\maketitle

\section*{Enoncé}
Calculer la probabilité que $k$ nombres soient premiers entre-eux, en utilisant la fonction de Mobïus.
\section*{Solution}
Introduisons $c_n^k = \{(a_1,\ldots,a_k)\in\llbracket 1;n\rrbracket ^k : a_1 \wedge \ldots \wedge a_k = 1\}$.
Posons $n = p_1\ldots p_v$, puis pour tout $i\in\llbracket 1;v\rrbracket$,
\begin{align*}
A_i = \{j\in\llbracket 1;n\rrbracket : p_i | j\}.
\end{align*}

Alors $c_n^k = \overline{\bigcup_{i=1}^{v}A_i^k}$, d'où
\begin{align*}
|c_n^k| &= n^k -\sum_{\varnothing\neq I\subset \llbracket 1;v\rrbracket}(-1)^{|I|-1}\bigg|\bigcap_{i\in I}{A_i^k}\bigg|\\
&= n^k+\sum_{\varnothing\neq I\subset \llbracket 1;v\rrbracket}(-1)^{|I|}\bigg \lfloor \frac{n}{\prod_{i\in I}p_i}\bigg\rfloor^k \\
&= \sum_{d=1}^n \mu(d)\lfloor n/d\rfloor^k 
\end{align*}

Ainsi \begin{align*}
|c_n^k|/n^k \longrightarrow &\quad \sum_{d=1}^{\infty}{\mu(d)/d^k}\\
& = \prod_{p\in \mathbb{P}}\bigg\{\sum_{j=0}^{\infty}\mu(p^j)p^{-jk}\bigg\}\\
& = \prod_{p\in \mathbb{P}}1-p^{-k} \\ &= \frac{1}{\zeta(s)}
\end{align*}

En utilisant l'identité du produit Eulérien (voir ?) puis $\mu(p) = -1$, $\mu(1) = 1$, $\mu(p^j) = 0$ pour $j\geq 2$.
\QED
\end{document}