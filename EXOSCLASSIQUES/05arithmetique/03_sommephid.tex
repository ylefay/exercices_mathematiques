\documentclass{article}
\usepackage[utf8]{inputenc}
\usepackage{amsmath}
\usepackage{amsmath}
\usepackage{amssymb}
\usepackage{french}
\usepackage{stmaryrd}
\usepackage{geometry}
\geometry{hmargin=2.5cm,vmargin=1.5cm}
\newcommand*{\QED}{\hfill\ensuremath{\blacksquare}}%
\begin{document}
\title{FICHE 05-03 : $\sum_{d|n}\phi(d) = n$ msc}
\author{Yvann Le Fay}
\date{Juin 2019}
\maketitle

\section*{Enoncé}
Montrer que $\sum_{d|n}\phi(d) = n$.
\section*{Solution}
L'égalité provient de la partition suivante
\begin{align*}
\llbracket 1;n\rrbracket = \bigsqcup_{d|n}\{k\in\llbracket 1;d\rrbracket : (k,d) = 1\}.
\end{align*}

Pour la justifier, considérons les fractions $\frac{1}{n}$, $\frac{2}{n}, \ldots$, $\frac{n}{n}$. Celles-ci sont après réduction, de la forme $\frac{a}{d}$ avec $d\mid n$, $a\wedge d = 1$ et $a\in\llbracket 1;d\rrbracket$, d'où la partition. On en déduit l'égalité sur les cardinaux.
\QED
\end{document}