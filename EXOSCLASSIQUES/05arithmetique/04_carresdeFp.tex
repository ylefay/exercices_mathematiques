% vim:ft=tex:
%
\documentclass{article}
\usepackage[utf8]{inputenc}
\usepackage{amsmath}
\usepackage{amssymb}
\usepackage{french}
\usepackage{stmaryrd}
\usepackage{geometry}
\geometry{hmargin=2.5cm,vmargin=1.5cm}
\newcommand*{\QED}{\hfill\ensuremath{\blacksquare}}%
\DeclareMathOperator{\im}{im}
\begin{document}
\title{FICHE 05-04 : Sur les carrés de $\mathbb{F}_p$.}
\author{Yvann Le Fay}
\date{Août 2019}
\maketitle
\section*{Enoncé}
Dénombrer les carrés de $\mathbb{F}_p$ puis démontrer que $\forall x \in \mathbb{F}_p^*,\,\,\, x  \textup{ est un carré} \Longleftrightarrow x^{\frac{p-1}{2}} = 1$.
\section*{Solution}
L'ensemble des carrés de $\mathbb{F}_p^*$ est l'image de $\mathbb{F}_p^*$ par l'application $f : x \mapsto x^2$ qui est un morphisme. Ainsi, on sait que $\im f \equiv \mathbb{F}_p^*/\ker f$. Or $\ker f = \{-1, 1\}$, ainsi, il y a exactement $\frac{p-1}{2}$ carrés dans $\mathbb{F}_p$.

D'après le petit théorème de Fermat, tout élément $x$ de $\mathbb{F}_p^*$ est solution de $x^{p-1} - 1 = 0$. Or $x^{p-1}  - 1 = (x^{\frac{p-1}{2}}-1)(x^{\frac{p-1}{2}}+1)$. On vérifie bien que $\forall x \in \mathbb{F}_p^*,\,\,\, (x^2)^{\frac{p-1}{2}} = 1$, et de plus, le premier membre de la factorisation égalisé à $0$, a dans $\mathbb{F}_p^*$ au plus $\frac{p-1}{2}$ solutions. Nécessairement, ce sont les carrés de $\mathbb{F}_p^*$. Or aucun de ces carrés n'est solution de l'équation du second membre de la factorisation. De plus nécessairement par le petit théorème de Fermat, $\forall x \in \mathbb{F}_p^*, x^{\frac{p-1}{2}} \in \{-1, 1\}$. Ainsi, les non-carrés de $\mathbb{F}_p^*$ sont exactement les solutions du second membre de la factorisation. 

Ce qui permet de conclure, que $x$ est un carré (ou non) dans $\mathbb{F}_p^*$ est équivalent à $x^{\frac{p-1}{2}} = 1$ (ou $-1$) respectivement.

On a par exemple que si ni $2$ ni $3$ ne sont des carrés dans $\mathbb{F}_p$, alors $6$ en est forcément un.
\QED
\end{document}
