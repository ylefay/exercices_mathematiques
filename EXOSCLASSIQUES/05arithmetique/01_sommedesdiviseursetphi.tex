\documentclass{article}
\usepackage[utf8]{inputenc}
\usepackage{amsmath}
\usepackage{amsmath}
\usepackage{amssymb}
\usepackage{french}
\usepackage{stmaryrd}
\usepackage{geometry}
\geometry{hmargin=2.5cm,vmargin=1.5cm}
\newcommand*{\QED}{\hfill\ensuremath{\blacksquare}}%
\begin{document}
\title{FICHE 05-01 : Somme des diviseurs et indicatrice d'Euler ALG? K-12-1-14:18}
\author{Yvann Le Fay}
\date{Juin 2019}
\maketitle

\section*{Enoncé}
Calculer $\sigma(n)$ la somme des diviseurs de $n\in \mathbb{N}$, majorer $\sigma(n)$. Démontrer que $\varphi(n) = \displaystyle n\prod_{p|n}\bigg(1-\frac{1}{p}\bigg)$.
\section*{Solution}
Considérons $n=p_1^{\alpha_1}\ldots p_k^{\alpha_k}$, alors
\begin{align*}
\sigma(n) &= \sum_{(\beta_j)\in \prod_{j=0}^{k}\llbracket 0;\alpha_j\rrbracket}\prod_{i=1}^k p_i^{\beta_i}\\
&=\prod_{i=1}^k \sum_{\beta_i\in\llbracket 0;\alpha_i\rrbracket}{p_i^{\beta_i}}\\
&=\prod_{i=1}^k \frac{p_i^{\alpha_i+1}-1}{p_i-1}
\end{align*}

Il est clair que l'ensemble des diviseurs de $n$ est compris dans $\{\frac{n}{k} : k\in\llbracket 1;n\rrbracket\}$, ainsi, 
\begin{align*}
\sigma(n)\leq n\sum_{k=1}^n \frac{1}{k}\sim n \ln{n}
\end{align*}

On peut utiliser le caractère multiplicatif de $\varphi$ ou le théorème des restes chinois, les deux mènent au résultat, on utilise le théorème des restes chinois. 
\begin{align*}
\varphi(n) &= |(\mathbb{Z}/{n\mathbb{Z}})^*|\\
&= \bigg|\prod_{p|n}\bigg(\mathbb{Z}/{p^{v_p(n)}\mathbb{Z}}\bigg)^*\bigg|\\
&=\prod_{p|n}\bigg|\bigg(\mathbb{Z}/{p^{v_p(n)}\mathbb{Z}}\bigg)^*\bigg|\\
&=\prod_{p|n}p^{v_p(n)}-p^{v_p(n)-1}\\
&=n\prod_{p|n}\bigg(1-\frac{1}{p}\bigg)
\end{align*}

On peut justifier que pour toute puissance première, $p^a$, on a $|\mathbb{Z}/{p^a\mathbb{Z}}| = p^a - p^{a-1}$ par un argument de dénombrement. Les éléments qui ne sont pas inversibles dans $\mathbb{Z}/{p^a\mathbb{Z}}$ sont ceux qui sont divisibles par $p$ et qui sont compris entre $1$ et $p^a$, autrement dit, l'ensemble des non inversibles est $\{kp : k\in\llbracket 1;p^{a-1}\rrbracket\}$, un passage au complémentaire fournit le résultat.
\QED
\end{document}
