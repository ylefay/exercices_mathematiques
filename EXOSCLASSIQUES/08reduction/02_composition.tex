\documentclass{article}
\usepackage[utf8]{inputenc}
\usepackage{amsmath}
\usepackage{amsmath}
\usepackage{amssymb}
\usepackage{french}
\usepackage{stmaryrd}
\usepackage{geometry}
\geometry{hmargin=2.5cm,vmargin=1.5cm}
\newcommand*{\QED}{\hfill\ensuremath{\blacksquare}}%
\DeclareMathOperator{\im}{im}
\DeclareMathOperator{\diag}{diag}
\DeclareMathOperator{\tr}{Tr}
\DeclareMathOperator{\Sp}{Sp}
\begin{document}
\title{FICHE 08-02 : Sur la composition}
\author{Yvann Le Fay}
\date{Novembre 2019}
\maketitle
\section*{Enoncé}
Soit $g \in \mathcal{L}(E)$ avec $E$ de dimension finie. Soit $\varphi(g) : f \mapsto f\circ g$. Pour tout $\lambda\in \Sp g$, calculer $\dim E_{\lambda}(\varphi(g))$. Montrer que $g$ est diagonalisable si et seulement si $\varphi(g)$ l'est.
\section*{Solution}
On a 
\begin{align*}
	f \in E_{\lambda}(\varphi(g)) \Longleftrightarrow g \circ f = \lambda f \Longleftrightarrow \im f \subset E_{\lambda}(g)
\end{align*}

On en déduit donc que $\dim E_{\lambda}(\varphi(g)) = n\dim E_{\lambda}(g)$ et ainsi $n^2 = \sum_{\lambda\in \Sp\varphi(g)} \dim E_{\lambda}\varphi(g) \Longleftrightarrow n = \sum_{\lambda \in \Sp g} \dim E_{\lambda}g$, i.e $\varphi(g)$ est diagonalisable si et seulement si $g$ l'est. 
\QED
\end{document}
