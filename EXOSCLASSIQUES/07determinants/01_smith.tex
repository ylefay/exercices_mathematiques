\documentclass{article}
\usepackage[utf8]{inputenc}
\usepackage{amsmath}
\usepackage{amsmath}
\usepackage{amssymb}
\usepackage{french}
\usepackage{stmaryrd}
\usepackage{geometry}
\geometry{hmargin=2.5cm,vmargin=1.5cm}
\newcommand*{\QED}{\hfill\ensuremath{\blacksquare}}%
\DeclareMathOperator{\im}{im}
\DeclareMathOperator{\diag}{diag}
\begin{document}
\title{FICHE 07-01 : Déterminant de Smith MET-1 3.6.6}
\author{Yvann Le Fay}
\date{Juillet 2019}
\maketitle
\section*{Enoncé}
Soit $n\in\mathbb{N}$, $\psi : \mathbb{N}^* \mapsto \mathbb{C}$, on définit pour tout $i,\,j\in\llbracket 1;n\rrbracket$, $a_{i,j}=\sum_{k|i,\,k|j}\psi(k)$ et $A=(a_{i,j})$. Montrer que
\begin{align*}
\det A =\prod_{k=1}^n \psi(k)
\end{align*}

Appliquer cette formule pour calculer $\det A$ où $a_{i,j}$ est respectivement, le nombre de diviseurs communs à $i$ et à $j$, la somme des diviseurs communs à $i$ et à $j$, $i\wedge j$.
\section*{Solution}
Posons $b_{i,j} = 0$ si $i\nmid j$ et $b_{i,j}=1$ so $i\mid j$, puis $B=(b_{i,j})$, alors,
\begin{align*}
a_{i,j}=\sum_{k|i,\,k|j}{\psi(k)}=\sum_{k=1}^n b_{k,j}b_{k,i}\psi(k)
\end{align*}

D'où, $A=\,^{t}B\diag(\psi(k))B$ puis $\det A = \prod_{k=1}^{n}\psi(k)$.

Pour le premier cas on pose $\psi(k)=1$ et on trouve $1$, le second, $\psi(k)=k$ et on trouve $n!$ et le troisième on utilise,
\begin{align*}
n = \sum_{k|n}\varphi{k}
\end{align*}

où $\varphi$ est l'indicatrice d'Euler, on utilise donc $n=i \wedge j$, d'où on tire
\begin{align*}
i\wedge j =\sum_{k\mid i\wedge j}\varphi(k) = \sum_{k\mid i,\,k\mid j}\varphi(k)
\end{align*}

puis $\det A = \prod_{k=1}^n \varphi(k)$.
\QED
\end{document}
