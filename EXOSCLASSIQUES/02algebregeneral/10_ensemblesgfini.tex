\documentclass{article}
\usepackage[utf8]{inputenc}
\usepackage{amsmath}
\usepackage{amsmath}
\usepackage{amssymb}
\usepackage{french}
\usepackage{stmaryrd}
\usepackage{geometry}
\geometry{hmargin=2.5cm,vmargin=1.5cm}
\newcommand*{\QED}{\hfill\ensuremath{\blacksquare}}%
\DeclareMathOperator{\im}{im}
\begin{document}
\title{FICHE 02-10 : Groupe dont l'ensemble des sous-groupes est fini: ALG1-01 2.2}
\author{Yvann Le Fay}
\date{Juillet 2019}
\maketitle
\section*{Enoncé}
Caractériser les groupes dont l'ensemble des sous-groupes est fini.
\section*{Solution}
Soit $G$ un groupe dont l'ensemble $E$ des sous-groupes est fini. Soit $g\in G$, alors $g$ est d'ordre fini car sinon $\langle g \rangle$ est isomorphe à $\mathbb{Z}$ qui admet une infinité de sous-groupes. De plus si $E'$ dénote le sous ensemble de $E$ composé des groupes monogènes de $G$ alors $G = \bigcup_{H\in E'} H$, or $E'$ est fini. $G$ est donc fini en tant qu'union finie de groupes finis. Le résultat réciproque est évident. 
\QED
\end{document}