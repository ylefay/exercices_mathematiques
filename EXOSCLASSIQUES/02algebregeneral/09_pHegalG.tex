\documentclass{article}
\usepackage[utf8]{inputenc}
\usepackage{amsmath}
\usepackage{amsmath}
\usepackage{amssymb}
\usepackage{french}
\usepackage{stmaryrd}
\usepackage{geometry}
\geometry{hmargin=2.5cm,vmargin=1.5cm}
\newcommand*{\QED}{\hfill\ensuremath{\blacksquare}}%
\DeclareMathOperator{\im}{im}
\begin{document}
\title{FICHE 02-09 : $p|H|=|G|$: MET-1 1.4.8}
\author{Yvann Le Fay}
\date{Juillet 2019}
\maketitle
\section*{Enoncé}
Soit $G$ un groupe fini et $H$ un sous groupe tel que $p|H|=|G|$ où $p$ est le plus petit diviseur premier de $|G|$, montrer que $H$ est distingué dans $G$.
\section*{Solution}
Posons la relation d'équivalence $\sim$,
\begin{align*}
\forall x,\, y\in G,\quad x\sim y \Longleftrightarrow x^{-1}y\in H
\end{align*}

Les classes d'équivalences sont les classes à gauche, posons $X=G\!\raisebox{-.50ex}{\ensuremath{/ \sim}}$, alors $|X|=p$ pour les mêmes raisons que dans la démonstration du théorème de Lagrange, et
\begin{align*}
\sigma_g : \left\{
     \begin{array}{@{}l@{\thinspace}l}
     X&\to X\\
     x&\mapsto gx
     \end{array}\right. \quad \varphi : \left\{
     \begin{array}{@{}l@{\thinspace}l}
     G&\to \mathfrak{S}_X\\
     g&\mapsto \sigma_g
     \end{array}\right. 
\end{align*}

On peut vérifier que $\sigma_g$ a pour réciproque $\sigma_{g^{-1}}$ et que c'est donc un élément de $\mathfrak{S}_X$. Par le premier théorème d'isomorphie, on a
\begin{align*}
\im \varphi \cong G\!\raisebox{-.50ex}{\ensuremath{/ \ker \varphi}}< \mathfrak{S}_X
\end{align*}

Puis le théorème de Lagrange et la condition sur $p$ donnent $\frac{|G|}{|\ker \varphi|}\mid p!$ puis $\frac{|G|}{|\ker \varphi|}\mid p$, donc 
\begin{align*}
|\ker \varphi|\geq |G|/p=|H|
\end{align*} 

De plus, on obtient que 
\begin{align*}
\ker \varphi =\{g\in G :\forall x\in G xgx^{-1}\in H\}\subset H
\end{align*}

Ce qui permet d'affirmer que $\ker \varphi = H$ et $H$ est distingué dans $G$.
\QED
\end{document}